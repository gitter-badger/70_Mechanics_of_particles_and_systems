\documentclass[12pt]{article}
\usepackage{pmmeta}
\pmcanonicalname{FirstOrderOperatorsInRiemannianGeometry}
\pmcreated{2013-03-22 15:28:18}
\pmmodified{2013-03-22 15:28:18}
\pmowner{rmilson}{146}
\pmmodifier{rmilson}{146}
\pmtitle{first order operators in Riemannian geometry}
\pmrecord{7}{37325}
\pmprivacy{1}
\pmauthor{rmilson}{146}
\pmtype{Definition}
\pmcomment{trigger rebuild}
\pmclassification{msc}{70G45}
\pmclassification{msc}{53B20}
\pmrelated{Gradient}
\pmrelated{Curl}
\pmrelated{Divergence}
\pmrelated{LeibnizNotationForVectorFields}

\endmetadata

\usepackage{amsmath}
\usepackage{amsfonts}
\usepackage{amssymb}


\newcommand{\fM}{\mathcal{C}^{\infty}(M)}
\newcommand{\vfM}{\mathcal{X}(M)}
\newcommand{\ddx}[1]{\frac{\partial}{\partial x^{#1}}}
\newcommand{\ddxf}[2]{\frac{\partial #2}{\partial x^{#1}}}

\newcommand{\grad}{\operatorname{grad}}

\newcommand{\iprod}{\mathop{\rfloor}}

\newcommand{\bg}{\mathbf{g}}
\newcommand{\Div}{\operatorname{div}}
\newcommand{\curl}{\operatorname{curl}}
\begin{document}
On a pseudo-Riemannian manifold $M$, and in Euclidean space in
particular, one can express the gradient operator, the divergence
operator, and the curl operator (which makes sense only if $M$ is
3-dimensional) in terms of the exterior derivative.  Let $\fM$ denote
the ring of smooth functions on $M$; let $\vfM$ denote the
$\fM$-module of smooth vector fields, and let $\Omega^1(M)$ denote the
$\fM$-module of smooth 1-forms. The contraction with the metric tensor
$g$ and its inverse $g^{-1}$, respectively, defines the
$\fM$-module isomorphisms
\[\flat:\vfM\to\Omega^1(M),\quad \sharp \colon \Omega^1(M)\to\vfM.\]
In local coordinates, this isomorphisms is expressed as
\[
\left(\ddx{i}\right)^\flat= \sum_j g_{ij} dx^j,\quad 
\left(dx^j\right)^\sharp = \sum_i g^{ij} \ddx{i}.
\]
or as the lowering of an index.  To wit, for $V\in\vfM$, we have
\begin{align*}
V&=\sum_{i=1}^n V^i\ddx{i},\\
(V^\flat)_j &= \sum_{i=1}^n g_{ij} V^i,\quad j=1,\ldots,n.\\
\end{align*}

The gradient operator, which in tensor notation is expressed as
$$(\grad f)^i = g^{ij} \ddxf{j}{f},\quad f\in \fM,$$
can now be defined as
$$\grad f = (df)^\sharp,\quad f \in\fM.$$

Another natural structure on an $n$-dimensional Riemannian manifold is
the volume form, $\omega \in \Omega^n(M)$, defined by $$\omega =
\sqrt{\det g_{ij}}\, dx^1\wedge\ldots\wedge dx^n.$$
Multiplication by
the volume form defines a natural isomorphism between functions and
$n$-forms: $$f\mapsto f\omega,\quad f\in \fM.$$
Contraction with the
volume form defines a natural isomorphism between vector fields and
$(n-1)$-forms: $$X\mapsto X\iprod \omega,\quad X\in \vfM,$$
or
equivalently $$\ddx{i} \mapsto (-1)^{i+1} \sqrt{\det g_{ij}}\,
dx^1\wedge\ldots \wedge \widehat{dx^i}\wedge\ldots \wedge dx^n,$$
where $\widehat{dx^i}$ indicates an omitted factor. The divergence
operator, which in tensor notation is expressed as $$\Div X = \nabla_i
X^i,\quad X\in \vfM$$
can be defined in a coordinate-free way by the following relation:
$$(\Div X)\, \omega = d(X\iprod \omega),\quad X\in \vfM.$$


Finally, on a $3$-dimensional manifold we may define the curl
operator in a coordinate-free fashion by means of the following relation:
$$(\curl X)\iprod \omega = d(X^\flat),\quad X\in \vfM.$$
%%%%%
%%%%%
\end{document}
