\documentclass[12pt]{article}
\usepackage{pmmeta}
\pmcanonicalname{ParticleMovingOnACardioidAtConstantFrequency}
\pmcreated{2013-03-22 17:14:20}
\pmmodified{2013-03-22 17:14:20}
\pmowner{perucho}{2192}
\pmmodifier{perucho}{2192}
\pmtitle{particle moving on a cardioid at constant frequency}
\pmrecord{6}{39568}
\pmprivacy{1}
\pmauthor{perucho}{2192}
\pmtype{Topic}
\pmcomment{trigger rebuild}
\pmclassification{msc}{70B05}

% this is the default PlanetMath preamble.  as your knowledge
% of TeX increases, you will probably want to edit this, but
% it should be fine as is for beginners.

% almost certainly you want these
\usepackage{amssymb}
\usepackage{amsmath}
\usepackage{amsfonts}

% used for TeXing text within eps files
%\usepackage{psfrag}
% need this for including graphics (\includegraphics)
%\usepackage{graphicx}
% for neatly defining theorems and propositions
%\usepackage{amsthm}
% making logically defined graphics
%%%\usepackage{xypic}

% there are many more packages, add them here as you need them

% define commands here
\newtheorem{theorem}{Theorem}
\newtheorem{defn}{Definition}
\newtheorem{prop}{Proposition}
\newtheorem{lemma}{Lemma}
\newtheorem{cor}{Corollary}

\begin{document}
This is another elementary example{\footnote{C.F. particle moving on the astroid at constant frequency}} about particle kinematics. In this case we will use polar coordinates. Let us consider the cardioid {\footnote{the locus of the points of the plane described by a circle (or disc) boundary point which it  is rolling over another one with the same radius $R$.}}
$$r=4R\cos^2\frac{\omega t}{2},$$ 
{\footnote{indeed the native polar equation of the cardioid is
$$r=2R(1+\cos\theta), \quad \theta=\omega t.$$ }}
with $R,\omega>0$ given constants and $t\in [0,\infty)$ means time parameter. The position vector of a particle, respect to an orthonormal reference basis $\{\mathbf{\hat{r}},\mathbf{\hat{\theta}}\},$ moving on the cardioid is
$$\mathbf{r}=4R\cos^2\frac{\omega t}{2}\,\mathbf{\hat{r}},$$
and its velocity {\footnote{in polar coordinates we have
$$\mathbf{\dot{r}}=\dot{r}\,\mathbf{\hat{r}}+r\dot{\theta}\, \mathbf{\hat{\theta}},$$
because the base vectors $\mathbf{\hat{r}}, \mathbf{\hat{\theta}}$ are changing on direction and sense according the formulas 
$$\frac{d\mathbf{\hat{r}}}{d\theta}=\mathbf{\hat{\theta}}, \qquad 
\frac{d\mathbf{\hat{\theta}}}{d\theta}=-\mathbf{\hat{r}}.$$
We are using the chain rule with $\dot{\theta}=\omega.$ Overdot denotes time differentiation everywhere.}}
$$\mathbf{v}=\mathbf{\dot{r}}=-4R\omega\sin\frac{\omega t}{2}\cos\frac{\omega t}{2}\,\mathbf{\hat{r}}+
4R\omega\cos^2\frac{\omega t}{2}\,\mathbf{\hat{\theta}}.$$
Therefore the speed is
$$v=4R\omega\cos\frac{\omega t}{2},$$
and the tangent vector
$$\mathbf{T}=-\sin\frac{\omega t}{2}\,\mathbf{\hat{r}}+\cos\frac{\omega t}{2}\,\mathbf{\hat{\theta}}.$$
Next we use the formula
$$\frac{v}{\rho}:=\big\Vert\mathbf{\dot{T}}\big\Vert=
\bigg\Vert -\frac{\omega}{2}\cos\frac{\omega t}{2}\,\mathbf{\hat{r}}
-\sin\frac{\omega t}{2}\,\mathbf{\dot{\hat{r}}}-
\frac{\omega}{2}\sin\frac{\omega t}{2}\,\mathbf{\hat{\theta}}+
\cos\frac{\omega t}{2}\,\mathbf{\dot{\hat{\theta}}}\bigg\Vert,$$
and by using the time derivative of base vectors
$$\frac{v}{\rho}=\bigg\Vert -\frac{3\omega}{2}\cos\frac{\omega t}{2}\,\mathbf{\hat{r}}-
\frac{3\omega}{2}\sin\frac{\omega t}{2}\,\mathbf{\hat{\theta}}\bigg\Vert,$$
getting the equation
$$v=\frac{3}{2}\omega\rho.$$



%%%%%
%%%%%
\end{document}
